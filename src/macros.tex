\newcommand{\mkAgenda}{
    \begin{frame}{Outline}
        \tableofcontents[currentsection]
    \end{frame}
}

\newcommand{\CUDA}[1]{{\mintinline[breaklines,breakafter=,]{CUDA}{#1}}}

\newcommand{\slide}[3]{
    \begin{frame}{#1}
        \center
        \begin{columns}
            \begin{column}{0.35\textwidth}
                \center
                \begin{itemize}
                    #2
                \end{itemize}
            \end{column}
            \begin{column}{0.6\textwidth}
                \begin{center}
                    #3
                \end{center}
            \end{column}
        \end{columns}
    \end{frame}
}

\newcommand{\append}[1]{
    \begin{minipage}[c]{0.66\linewidth}
    \resizebox{\textwidth}{!}{\input{#1}}
    \end{minipage}
}

\newcommand{\appendTwo}[2]{
    \begin{minipage}[c]{0.45\linewidth}
        \resizebox{\textwidth}{!}{\input{#1}}
    \end{minipage}
    \begin{minipage}[c]{0.45\linewidth}
        \resizebox{\textwidth}{!}{\input{#2}}
    \end{minipage}
}

\newcommand{\true}{TRUE}
\newcommand{\false}{FALSE}

\newcommand{\sectionSlide}[1]{
    \begin{frame}
      \vfill
      \centering
      \begin{beamercolorbox}[sep=8pt,center,shadow=true,rounded=true]{title}
        \usebeamerfont{title}\insertsectionhead\par%
      \end{beamercolorbox}
      \ifx#1\true{\tableofcontents[currentsection]}\fi
      \vfill
  \end{frame}
}